\chapter{Obtaining the Code}
\label{app:obtaincode}

All of the source code required to build the tools discussed in this paper, and the text of the paper itself,  is available on the author's github page at
\texttt{https://github.com/defreez}. There are nine relevant git repositories, all prefaced with the word ``thesis.''
A working Android ROM can be built for a Nexus S from the code above by following the directions at \texttt{source.android.com}, but
instead of initializing the tree against the manifest provided by Google, issue the command: 
\texttt{repo init -u https://github.com/defreez/thesis-ics-platform-manifest}

\begin{table}[!h]
\centering
\begin{tabular}{| l | p{3.2in} |}
\hline
\textbf{Repository} & \textbf{Description} \\
\hline
thesis &  The actual text of the document \\
\hline
thesis-ics-platform-manifest & The manifest which references all of the projects necesssary to build Android. \\
\hline
thesis-ics-platform-frameworks-base & Contains modifications at the framework level, such as calls to vold from within the
mount service. \\
\hline
thesis-ics-device-samsung-crespo & Includes the product configuration necessary for the build to work on a Nexus S \\
\hline
thesis-ics-platform-system-vold & The modifications to vold \\
\hline
thesis-ics-platform-system-core & Changes required for init to mount eCryptfs \\
\hline
thesis-ics-external-libecryptfs & eCryptfs support library \\
\hline
thesis-ics-external-libkeyutils & Keyutils support library \\
\hline
thesis-carve & Python scripts for carving YAFFS2 images \\
\hline
\end{tabular}
\end{table}

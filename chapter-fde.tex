\chapter{Android Evil Maid}
\label{ch:fde}

\emph{This chapter is adapted from an article originally posted on the author's personal website. It was posted there first to
provide a timely commentary on secure boot environments. It remains, as far as the author is aware, the only demonstrated evil maid
attack on Android \cite{androidevilmaid}.}

There is, it would seem, an obvious method of protecting data against the forensic inquiry outlined in chapter \ref{ch:forensics}.
Full Disk Encryption (FDE) is generally regarded as the best way to protect data in the event a device falls into the wrong hands,
be it thief or forensic investigator \cite{fdeworks}.  For Android, in particular, Whisper Systems has included FDE in their
security-oriented build WhisperCore \cite{whispercore}, and Google itself has included FDE in Android 3.x and 4.x. This chapter
focuses on the implementation of FDE provided in WhisperCore rather than the native Android encryption scheme for a couple of
reasons.  First, the ability to modify Android is at the heart of this paper, and WhisperCore is the only popular FDE implementation
for Android provided by aftermarket modifications to the OS.  Partially prompted by Twitter's purchase of Whisper Systems and the
subsequent unavailability of WhisperCore, Chris Soghoian once commented to Ars Technica:
\begin{quote} 
\ldots these applications that people are creating, that activists are creating, and then abandoning six months after their funding
runs out -- those are just a waste of time. Those are never going to go anywhere and they're never going to be used by anyone. We
need technologies that can be used by millions of consumers, without playing with configuration options. 

\hspace{\fill}\cite{arstechnica}
\end{quote}
This is a sentiment that has some truth to it, but it is the opinion of the author that their remains value in
developing privacy-preserving tools, even if they are not adopted by Google.  Whisper Systems greatly increased awareness of significant
security issues on the Android platform, and offered full disk encryption on Android before anyone else, including Google. Second,
WhisperCore on the Nexus One is vulnerable to the evil maid attack described here, which highlights the fact that FDE can only be
relied on so long as an attacker has not had physical access to the device. It demonstrates in a very real way the dangers of
leaving a device unguarded. 

As powerful as disk encryption is, it does not provide a complete privacy solution. The problem is that the data is
\emph{still there}.  The scenarios which form the basis of this project, posit that physical control of the device has been lost.
Once that is the case, it becomes difficult to ensure the integrity of the decryption mechanism, exposing the target to the
possibility that forensic artifacts which have been accumulating over time could be retrieved.  Furthermore, the use of force could
compel the target to divulge the decryption key, again exposing accumulated artifacts. 

Chapter \ref{ch:introduction} asserted that three conditions were necessary for encryption to succeed as a privacy-preserving tool.
\begin{enumerate}
	\item{The encryption passphrase must be strong}
	\item{An attacker cannot have access to the device before it used}
	\item{An attacker cannot have access to the device while it is running}
\end{enumerate}
The first item is mostly self-explanatory. If a phone is encrypted with the PIN ``1111'' then it does not matter how secure the
implementation is, decrypting the device will be easy. The third item will be discussed in chapter \ref{ch:ecryptfs}. A fourth
condition might be added, which is that the attacker is not willing to extract the key through force, what the security industry
likes to call ``rubber-hose'' cryptanalysis \cite{rubberhose}. It is the second condition that will now be taken up. 

\section{Evil Maid}

Encryption has been successful at protecting data at rest, but implementations often remain susceptible to a class of boot-time
attacks collectively referred to as "evil maid" attacks \cite{schneiermaid}. The typical scenario proposed for an evil maid attack
involves a person traveling with an encrypted computer. Typically the computer is a laptop, but fundamentally it does not matter
what the device is.  While the traveler is away from his or her room, an eponymous Evil Maid comes in to clean. The data on the
device cannot be read because it is encrypted, so the Evil Maid installs a boot-time keylogger. This keylogger could take any form,
hardware or software, but most often it takes the form of a small piece of software installed to the unencrypted boot portion of a
drive. When the traveler returns and decrypts the drive, the keylogger reads the FDE password as it is typed. The password could be
saved somewhere else on the drive for the Evil Maid to retrieve later, or immediately sent out over the network.
This is made possible by the fact that full disk encryption is something of a misnomer.  Entire partitions are often left
unencrypted if they do not contain user data.  The bootloader itself may be signed or encrypted, offering a degree of protection
against evil maid attacks, but doing so isn't foolproof \cite{attackingbitlocker} and comes at the price of locking down the
hardware. Moreover, even if the boot code itself is protected, a hardware of the device could be modified. There is, when it comes
down to it, absolutely no way to protect against an evil maid attack.  The reality of the situation is that for those truly
concerned about privacy, if an electronic device is ever out of their hands, it should be destroyed. There are simply too many ways
that an adversary might tamper with the device. This chapter describes one of them.

Against a desktop or laptop computer, evil maid attacks are typically implemented by booting from an alternate device, such as a
thumb drive in Joanna Rutkowska's attack against TrueCrypt \cite{evilmaid}.  Mobile devices offer natural protection against evil
maid attacks in the form of locked bootloaders.  There is no easy way to boot an Android phone or tablet from peripheral storage,
and it is usually either impossible to modify the OS image without an exploit or doing so requires wiping the device.
Unfortunately, the restrictive nature of the locking and unlocking mechanisms commonly impede utility by forcing the OS image to be
signed by the vendor, rather than by the owner/operator of the device. It will be demonstrated that WhisperCore on the Nexus One,
because it requires an unlocked bootloader, is inherently vulnerable to evil maid attacks.  WhisperCore on the Nexus S, however, is
comparatively resistant because the Nexus S bootloader can be relocked without reverting to stock images.  The reference tablet
architecture for Honeycomb is capable of deploying FDE without unlocking the bootloader, but only at a steep price: all OS images
remain entirely default and therefore all of the security advantages offered by distributions such as WhisperCore are forfeited.
Because the majority of bootloaders have been designed by phone manufacturers for their own purposes, end-users are required to
accept the level of security provided by the default OS build.  This situation is crippling the ability of firms to enhance the
platform and forcing the hobbyist community to accept inferior security.  At the time of writing, the rumored launch of the "Nexus
Prime," running Android 4.0 (Ice Cream Sandwich, or ICS) is only a few weeks away.  ICS will supposedly merge the phone and tablet
branches of Android.  Whether or not a Nexus S style bootloader is used will be telling for the future of Android security. 

\section{Attacking WhisperCore}
WhisperCore is a security-oriented Android distribution developed by Whisper Systems, which was founded by notable security
researcher Moxie Marlinspike.  WhisperCore is currently available for the flagship Android phones (Nexus One and Nexus S). In
addition to FDE, WhisperCore provides a significant number of security features, including a firewall, encrypted backups, and
encrypted voice and text messaging.  Some of these features can be used on a stock build of Android and are available in the Android
market, while others require the full WhisperCore build.  Furthermore, WhisperCore tends to receive important updates that other
Android builds overlook. When the Dutch certificate authority DigiNotar was compromised in 2011, for example, not only did Whisper
Systems remove the certificate, but they went so far as to extend the Android framework with a certificate blacklisting system to
prevent an SSL connection from any application using even an intermediate DigiNotar certificate \cite{whispernotar}. 

Deploying WhisperCore to a phone necessarily involves unlocking the bootloader.  The Nexus series of phones use the HBOOT
bootloader, which uses the fastboot protocol to communicate with a computer connected via USB.  Nexus phones are popular in the
Android community not only because they provide a "pure" Android experience, but because they have easily unlockable bootloaders.
To unlock a Nexus phone, simply connect it to a computer which has the fastboot utility installed and issue one command:
\texttt{fastboot oem unlock}.

Unlocking the bootloader wipes the phone and activates fastboot extended commands, enabling a user to flash custom images on the
phone.  The WhisperCore installer does this for the user automatically (after a warning).  WhisperCore flashes a custom system
partition containing the WhisperCore utilities, and a custom boot partition with a modified kernel and ramdisk.  On the Nexus S,
which supports bootloader relocking, the bootloader is then relocked.  On a Nexus S, the evil maid attack described here would be
non-trivial, as unlocking the bootloader to perform an attack would cause all data to be lost.  On the Nexus One, relocking is not
possible, leaving the device susceptible to evil maid attacks. 

\subsection{WhisperYAFFS}
WhisperCore includes two separate two FDE implementations, one for the Nexus One and one for the Nexus S. WhisperCore on the Nexus
One implements an encryption layer on top of YAFFS2, which is the native filesystem.  On the Nexus S, where ext4 is the native
filesystem, DM-Crypt (device-mapper crypto target) is used, which is a block-level encryption layer, rather than a cryptographic
filesystem (see chapter \ref{ch:ecryptfs} for more on the difference between block-level an file-level encryption).  DM-Crypt is the
same encryption mechanism that Google uses for Android devices that support FDE, and often how Linux desktops are encrypted.  The
rest of this section will focus on WhisperCore as it exists running on a Nexus One phone.

YAFFS2 \cite{howyaffsworks}, or Yet Another Flash Filesystem version 2, does not normally support encryption.  Though Android
appears to be moving toward \texttt{ext4} on top of hardware wear-leveling, the vast majority of devices running Android still use
YAFFS2.  In order to provide encryption for devices using YAFFS2, Whisper Systems, impressively, developed an extension to YAFFS2
providing AES encryption at the page level.  This extended version of YAFFS2 is branded WhisperYAFFS and, unlike the rest of
WhisperCore, is released as open source software (likely by necessity, as YAFFS2 is distributed under the GPL unless a special
license is obtained from Aleph One).

Because the source code for WhisperYAFFS is freely available \cite{whisperyaffs}, building an Android kernel with WhisperYAFFS
support is relatively straight-forward. The kernel for the Nexus One is contained in the kernel/msm.git project, which contains the
Android kernel that runs on Qualcomm chipsets.  The WhisperYAFFS code is a full replacement for YAFFS2 in the kernel; in the kernel
source, the entire \texttt{fs/yaffs2} directory can be removed and replaced with the WhisperYAFFS source.  For our purposes, the most
interesting portion of WhisperYAFFS to examine is in \texttt{yaffs\_vfs\_glue.c}.  This is where WhisperYAFFS (and YAFFS2 before it)
interfaces with Linux VFS to register the filesystem and mount devices.  It turns out that the password used to mount an encrypted
WhisperYAFFS filesystem is passed as a mount option.  WhisperYAFFS has actually added two mount options, \texttt{unlock\_encrypted=}
and \texttt{create\_encrypted=}, which perform self-evident functions.  The password is passed after the equals sign, parsed into a
struct, and then either the unlock or create functions are called.  During the boot process, a modified \texttt{init.mahimahi.rc} is
used to to call a binary on the system partition that presents a minimal UI for password entry and makes the appropriate mount call. 

\subsection{Switching Kernels}
In order to perform the evil maid attack, a custom kernel will be built that can run WhisperCore, but has the unlock function hooked
to store the decryption password.  The entire kernel source for WhisperCore is not available, just WhisperYAFFS, so for the sanctity
of the GPL hopefully nothing else has changed.  When attempting to blindly replace a kernel for a target system, it is a good idea
to replicate the kernel config, which is available at \texttt{/proc/config.gz}.  In the example code published at
\texttt{https://github.com/defreez/android-kernel-msm}, this config is stored at \texttt{arch/arm/configs/whisper\_defconfig}.  The
\texttt{build-kernel.sh} shell script provided is a tweaked version of the script accompanying the ``goldfish'' (emulator) version
of the kernel, which conveniently uses the Android build system for cross-compilation.  Pulling the config and building it with
WhisperYAFFS is sufficient to boot WhisperCore.

In order to retrieve the unlock password, a little bit of code is tacked on to the \texttt{yaffs\_UnlockEncryptedFilesystem}
function in \texttt{yaffs\_vfs\_glue.c}. A section is added which reads the password as the user enters it and writes it writes it
out to a file on the unencrypted system partition.  The SD card would have been the first choice for a location to store the
password, but WhisperCore contains an option to encrypt the SD card as well. The system partition is normally read-only, so it is
first remounted read-write. The password is written to /system/etc/em.txt, though the location is arbitrary. After the password has
been written, the system partition is remounted RO. Any number of different approaches could be taken to write the password out to a
more subtle location that could be read later, perhaps in an area out of band from the normal filesystem, or to even send the
password out over the network. 

\begin{table}
\lstinputlisting{tables/evilmaid-storepass.c}
\label{tab:storepass}
\caption{Evil Maid Patch: Store WhisperYAFFS Unlock Password}
\end{table}

The system partition, while a conveniently accessible location for writing during unlock, cannot be read offline.  When the
hypothetical evil maid returns to collect the device, \texttt{/system} will not be available because the device does not have USB
debugging enabled.  How then, does the evil maid retrieve the password? Hardcode a backdoor.  If the user enters "evilmaid" as the
password, then read the saved password from disk.  The partition is decrypted and the phone boots normally. 

\begin{table}
\lstinputlisting{tables/evilmaid-backdoor.c}
\label{tab:backdoor}
\caption{Evil Maid Patch: Backdoor}
\end{table}

Finally, switching out the kernel will change the ``vermagic'' string.\footnote{The vermagic string identified the kernel version
that a kernel module was compile for.} The vermagic string must match for kernel modules to load.  In the Nexus One, wireless is
provided by the \texttt{bcm4329.ko} kernel module, thus if the vermagic string changes wireless will not work.  Breaking wireless is likely
to tip off the target, so in the evil maid kernel the vermagic string has been hardcoded in the Makefile to match WhisperCore. 
The compile version can also be hardcoded by preventing \texttt{scripts/mkcompile\_h} from overwriting \texttt{compile.h}. 
The advantage of doing so is that the "Kernel version" visible from the "About phone" settings menu will be indistinguishable from WhisperCore.

Applying the evil maid kernel to the phone requires repacking a boot image.  There is a small utility in the \texttt{exbootimg}
directory of the evil maid kernel source that, given the correct path to a boot image, will extract the kernel and ramdisk.  The
utility is a quick and dirty modification of \texttt{mkbootimg} that does not do much error checking.  The WhisperCore download
contains a boot.img.  Simply swap out the kernel and repack with the mkbootimg utility that is built as part of the Android Open
Source Project.  The \texttt{mkbootimg} incantation is: \texttt{mkbootimg --kernel EVILMAIDKERNEL --ramdisk EXRAMDISK -o
/tmp/evilmaid.img --base 0x20000000}.  This boot image can be flashed to the phone and will save the decryption password as the disk
is unlocked.  A prebuilt boot image for WhisperCore 0.5.5 running on a Nexus One is available at
\url{http://www.defreez.com/blob/android-evilmaid-whisper-0.5.5_r0.img}.

To recap, the following steps will successfully perform an evil maid attack against WhisperCore on a Nexus One:

\vbox{
	\begin{enumerate}
		\item{Obtain the evil kernel.}
		\item{Build the kernel with \texttt{build-kernel.sh} (requires functioning Android build environment, and assumes
			the kernel is in a subdirectory of AOSP root).}
		\item{Unpack the WhisperCore boot image with exbootimg.}
		\item{Pack an evil maid boot image with \texttt{mkbootimg} using the evil maid kernel and the WhisperCore ramdisk.}
		\item{When the target is away, flash the new boot image.}
		\item{Wait for the target to decrypt the phone.}
		\item{Steal the phone.}
	\end{enumerate}
}

\section{Conclusion}
While the Android community has a tendency to ridicule locked bootloaders, securing the boot environment significantly mitigates
offline attacks against "Full Disk Encryption" (FDE).  Unfortunately, the methods typically used to secure the Android boot
environment have not taken into account the need to protect 3rd party images.  This is demonstrated by the differing resilience to
evil maid attacks exhibited by WhisperCore on the Nexus One and the Nexus S.  On the Nexus One, WhisperCore is incapable of
defending against an evil maid attack because the bootloader must remain unlocked.  On the Nexus S, which allows for the bootloader
to be relocked with custom images, an evil maid attack would not be easy to execute.  Attacks such as these, or simply the use of
force against the owner of the device, makes relying solely on FDE for privacy a perilous proposition.  This is not meant to
dissuade one against the use of encryption, quite the contrary. Disk encryption is the most powerful tool in a privacy advocate's
arsenal. Simply be aware that encryption is not panacea.

\chapter{Conclusion}

It is obvious that mobile devices are here to stay, and mobile forensics has already become one of the most significant areas of
computer security.  Mobile forensics is a legitimate field of study, with legitimate applications, and its methods are being
described widely. But the techniques can also be abused, and comparatively little research, is being done on privacy-preserving
techniques for mobile devices.  This is a situation that I, for one, would like to see changed. The purposes and applications of
such research vary widely, from hardening forensic methods to protecting privacy in the face of unjustified forensic inquiry.  This
study has attempted to provide insight into the technical basis for at least one privacy-preserving method applicable to Android
devices: encryption. As the field of Android forensics moves forward rapidly, it is crucial that the privacy community keep pace.
Already, live-system forensics is being introduced for Android devices, and soon live memory acquisition of phones and tablets will
be commonplace. For activists, dissidents, and plain-old citizens crossing borders, it is imperative that methods exist to protect
mobile data against these increasingly sophisticated acquisition techniques. This study has proposed one solution: a modification to
eCryptfs for Android that independently keys applications. Some of these keys can be wiped whenever the phone is locked, preventing
data from being copied off, even if the device is seized while it is running. Only time will tell if this is a good solution, and it
will require additional effort before it is fully resistant to memory analysis, but it is a reasonable attempt at a defense. There
are other projects working toward the same goal. By combining all of these solutions into a complete, privacy-oriented mobile OS,
maybe someday people will be free to use their devices they intend, and only as they intend, without fear of retribution.

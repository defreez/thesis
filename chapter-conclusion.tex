\chapter{Conclusion}

There is an uncomfortable tension between computer forensics and privacy. Computer forensics draws attention to all of the pieces of
information an individual leaves behind on a device.  Privacy-preserving technologies strive to hide, destroy, or otherwise prevent
the recovery of these same bits of information. A delicate balance between the two must be struck, as an individual may find
themselves requiring the use of either forensic tools or privacy-preserving technologies, depending on the situation. 
Android forensics is developing at an astonishing pace.  Already, live-system forensics is being introduced for Android devices, and
soon live memory acquisition of phones and tablets will be commonplace. For activists, dissidents, and plain-old citizens crossing
borders, it is imperative that methods exist to protect mobile data against these increasingly sophisticated acquisition techniques.
Toward that end, this study has focused on the privacy side of the equation.

This paper gave a short overview of Android forensics techniques, emphasizing the methods by which the data
on an Android device may be acquired. These methods explicitly attempt to subvert the privacy of data on a device by making a
low-level copy, and therefore serve as an excellent litmus test for privacy-preserving tools. Encrypting a device, so long as the
boot environment and the integrity of the hardware itself are protected, provides significant mitigation against the acquisition of
data. Traditional methods of encryption, however, do not provide significant protection of a device as it is running, particularly
if the attacker has a way of bypassing the lock screen.

This study has proposed a method of encryption capable of protecting a device even if it is seized while running.  The key
distinction between the method proposed - eCryptfs boundary mode - and traditional mobile full-disk encryption, is that eCryptfs
boundary mode independently keys Android applications to provide additional protection. The keys of select applications can be wiped
whenever the device is locked, while leaving the rest of the data accessible. This completely prevents logical acquisition of the
chosen applications, and could be extended to thwart live memory analysis. Coupling this technique with other projects
that provide secure communications, such as Orbot and TextSecure, would provide a privacy-oriented mobile OS capable of withstanding
significant forensic inquiry. Properly used, such a platform would afford individuals a much greater level of confidence that even
if they lose control of their device, their data is safe.

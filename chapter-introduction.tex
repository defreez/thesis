\chapter{Introduction}
\label{ch:introduction}

It has become apparent that the world is in the midst of a transition away from monolithic desktops toward mobile computing.  Two
giants, Google and Apple, currently dominate the mobile space, with Android and iOS respectively.  Google has marketed Android as
the more ``open" platform, while iOS remains a walled garden.  The open/closed distinction is dubious at best, as Android is still
developed behind closed doors.  Nonetheless, Google has released most of the source code for Android and done significantly less
than Apple to impede the efforts of the hacking community. For this reason, Android has become the platform of choice for developers
interested in modifying the mobile experience outside of the confines of a standard application.  This project has chosen to focus
on Android for the same reason.

While the headlines have been filled with the latest applications, processors, and screen sizes, the shift to mobile computing is also
bringing with it a deluge of security baggage. Mobile devices, like any other tool, can be used in a multitude of ways, and so comes
crime and surveillance, privacy invasion and preservation, in a word, forensics. The majority of forensic tools available for Android
devices are not open to inspection, and much of the detailed technical information available about Android forensic methods is
proprietary, though Andrew Hoog has broken some ground with his book \emph{Android Forensics} \cite{hoog}. This project is not an
attempt to comprehensively describe the field of Android forensics, nor is it an introduction to computer forensics. Rather, this is
a study in privacy preservation.

Research in computer forensics is commonly circumscribed by the needs of law and corporate policy enforcement.  For example, when
enumerating the various cases in which forensic analysis could be useful, Andrew Hoog includes civil and criminal investigations,
internal corporate investigations, family matters, and government security \cite{hoog}.  What these scenarios all have in common is
that the forensic techniques employed are all targeted toward providing information to investigators.  The legal right of the
investigator to process evidence is presumed.  Yet there are a wide range of scenarios, from the dissident hiding from a dictatorial
regime, to the prying eyes of a stalker, where the successful use of forensic techniques may inflict grave injustice upon the owner
of the object of analysis.  In the face of untoward forensic inquiry, a person has little recourse other than to prevent forensic
techniques from succeeding in the first place. This project is formed out of the realization that forensics can be used as a weapon.
Anyone concerned with the nature of digital privacy should have tools of their own. The goal is to lay bare the core forensic
techniques used to analyze Android devices, and then to demonstrate a relevant defense. It details the practical obstacles to
modifying an operating system consisting of millions of lines of code spread across hundreds of projects in over a dozen languages.
The goal is to provide a defense, even one that is narrowly conceived, which at least slows down forensic analysis and increases the
safety of data on an Android device. 

That defense, unsurprisingly, is encryption. Recent releases of Android include the option of full-disk encryption by default, which
works very well when certain conditions are met: the attacker cannot have physical access to the device prior to use, the encryption
passphrase must be strong, and the attacker cannot have privileged access to the device while it is running.  All of these
conditions are discussed later in the paper, but it is the last condition that is of particular concern. People simply do not turn
off their phones very often, which dramatically increases the likelihood that any forensic analysis performed on an Android device
will be performed on a \emph{running} Android device. The method of encryption introduced in this paper takes steps toward
addressing this issue, and demonstrates a technique that could dramatically increase the safety of data on a running device.

Chapter \ref{ch:forensics} introduces Android forensics. The chapter discusses how the data on an Android device is acquired. While
the analysis of acquired data is briefly discussed, the focus here is on the process of acquisition, because that is the critical
point at which data must be protected. That is to say, any data that has left the device unencrypted is lost to an attacker.  Next,
an existing method for securing Android against forensic inspection is discussed in chapter \ref{ch:fde}: full-disk encryption as
implemented by the third-party distribution of Android WhisperCore. Chapter \ref{ch:ecryptfs} provides the technical details of a
novel method of encrypting data on an Android device. It shows how the filesystem eCryptfs can be used on an Android device to
provide independent keys for each application, and how those keys can be wiped when the screen is locked. Finally, chapter
\ref{ch:future} discusses future directions for similar privacy research, as well as improvements that could be made to the current
encryption implementation.

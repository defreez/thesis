\chapter{Introduction}
\label{ch:introduction}

It has become apparent that the world is in the midst of a transition away from monolithic desktops toward mobile computing.  Two
giants, Google and Apple, currently dominate the mobile space, with Android and iOS respectively.  Google has marketed Android as
the more ``open" platform, while iOS remains a walled garden.  The open/closed distinction is dubious at best, though, as Android is
still developed behind closed doors.  Nonetheless, Google has released most of the source code for Android and done significantly
less than Apple to impede the efforts of the hacking community. For this reason, Android has become the platform of choice for
developers interested in modifying the mobile experience outside the confines of a standard application.  This study has chosen
to focus on Android for the same reason.

While the headlines have been filled with the latest applications, processors, and screen sizes, the shift to mobile computing is
also bringing with it a deluge of security baggage. Mobile devices are multi-purpose computing tools, and they often house very
important data. As with desktop systems, this invites crime and surveillance, privacy invasion and preservation, in a
word, forensics.  Unfortunately, the majority of forensic tools available for Android devices are not open to inspection, and much
of the detailed technical information is proprietary, though Andrew Hoog has broken some ground with his book \emph{Android
Forensics} \cite{hoog}.  This paper is not an attempt to comprehensively describe the field of Android forensics, nor is it an
introduction to computer forensics. Rather, this is a study of privacy on the Android platform that is informed by the field of
forensics.

Research in computer forensics is commonly circumscribed by the needs of law and corporate policy enforcement.  For example, when
enumerating the various cases in which forensic analysis could be useful, Andrew Hoog includes civil and criminal investigations,
internal corporate investigations, family matters, and government security \cite{hoog}.  What these scenarios all have in common is
that the forensic techniques employed are all targeted toward providing information to investigators.  The right of the investigator
to process evidence is presumed.  Yet there are a wide range of scenarios, from the dissident hiding from a dictatorial regime to
the prying eyes of a stalker, where the successful use of forensic techniques may inflict grave injustice upon the owner of the
object of analysis.  In the face of untoward forensic inquiry, a person has little recourse other than to prevent forensic
techniques from succeeding in the first place. For this reason, anyone concerned with the nature of digital privacy should have
tools of their own.  

The goal of this paper is to describe the forensic techniques used to acquire data from Android devices, and
then to demonstrate a relevant defense. It details the practical obstacles to modifying an operating system that consists of
millions of lines of code, spread across hundreds of projects in over a dozen languages.  The defense, unsurprisingly, is
encryption. Recent releases of Android include the option of full-disk encryption, which works very well when certain conditions are
met: the attacker cannot have physical access to the device prior to use, the encryption passphrase must be strong, and the attacker
cannot have access to the device while it is running.  All of these conditions are discussed later in the paper, but it
is the last condition that is of particular concern. People simply do not turn off their phones very often, which dramatically
increases the likelihood that any forensic analysis performed on an Android device will be performed on a \emph{running} Android
device. The method of encryption introduced in this paper takes a step toward addressing this issue, and demonstrates a technique
that increases the safety of data on a running device.

Chapter \ref{ch:forensics} introduces Android forensics. It discusses how data on an Android device can be acquired. While the
analysis of acquired data is briefly discussed, chapter \ref{ch:forensics} focuses on the process of acquisition because that is the
critical point at which data must be protected. Next, chapter \ref{ch:fde} discusses full-disk encryption in WhisperCore, a
third-party distribution of Android. The chapter uses an attack against WhisperCore to exemplify the perils of losing physical
control of a device. Chapter \ref{ch:ecryptfs} provides the technical details of a novel method of encrypting data on an Android
device, called eCryptfs boundary mode. It shows how the eCryptfs filesystem can be used to provide independent keys for each
application, and how those keys can be wiped when the screen is locked.  Finally, chapter \ref{ch:future} discusses future
directions for similar privacy research, as well as improvements that could be made to the implementation of eCryptfs boundary mode.

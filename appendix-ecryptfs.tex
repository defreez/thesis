\chapter{Patches to eCryptfs}
\begin{table}
\lstinputlisting{tables/kernel.h.patch}
\label{tab:ecryptfspatch-ecrytpfs_kernel.h}
\caption{Boundary Key Structures}
\end{table}
\begin{table}
\lstinputlisting{tables/crypto.c.patch1}
\label{tab:ecryptfspatch-crypto.c.patch1}
\caption{Boundary Key Generation}
\end{table}
\begin{table}
\lstinputlisting{tables/crypto.c.patch2}
\label{tab:ecryptfspatch-crypto.c.patch2}
\caption{Associate UID with inode crypto context}
\end{table}
\begin{table}
\lstinputlisting{tables/keystore.c.patch1}
\label{tab:ecryptfspatch-keystore.c.patch1}
\caption{Find Boundary Key in Keystore}
\end{table}
\begin{table}
\lstinputlisting{tables/keystore.c.patch3}
\label{tab:ecryptfspatch-keystore.c.patch3}
\caption{Encryption Routine}
\end{table}
\begin{table}
\lstinputlisting{tables/keystore.c.patch2}
\label{tab:ecryptfspatch-keystore.c.patch2}
\caption{Decryption Routine}
\end{table}
